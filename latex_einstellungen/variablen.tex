% Informationen ------------------------------------------------------------
% 	Definition von globalen Parametern, die im gesamten Dokument verwendet
% 	werden können (z.B auf dem Deckblatt etc.).
% --------------------------------------------------------------------------
\newcommand{\titel}{LaTeX Vorlage}
\newcommand{\untertitel}{Untertitel}
\newcommand{\art}{Art}
\newcommand{\fachgebiet}{Fachgebiet}
\newcommand{\autor}{Author}
\newcommand{\studienbereich}{Studienbereich}
\newcommand{\erstgutachter}{Erstgutachter}
\newcommand{\zweitgutachter}{-}
\newcommand{\matrikelnr}{Matrikelnummer}
\newcommand{\firma}{Firma}
\newcommand{\ort}{Ort}

\newcommand{\betreuer}{Betreuer}
\newcommand{\durchfuehrungsdatum}{Durchführungsdatum}
\newcommand{\durchfuehrungsort}{Durchführungsort}
\newcommand{\jahrgang}{Jahrgang}
\newcommand{\jahr}{Jahr}

% Eigene Befehle
\newcommand\tab[1][20pt]{\hspace*{#1}}

% Autorennamen in small caps
\newcommand{\AutorZ}[1]{\textsc{#1}}
\newcommand{\Autor}[1]{\AutorZ{\citeauthor{#1}}}

% Befehle zur semantischen Auszeichnung von Text
\newcommand{\NeuerBegriff}[1]{\textbf{#1}}
\newcommand{\Fachbegriff}[1]{\textit{#1}}
\newcommand{\Prozess}[1]{\textit{#1}}
\newcommand{\Webservice}[1]{\textit{#1}}
\newcommand{\Eingabe}[1]{\texttt{#1}}
\newcommand{\Code}[1]{\texttt{#1}}
\newcommand{\Datei}[1]{\texttt{#1}}
\newcommand{\Datentyp}[1]{\textsf{#1}}
\newcommand{\XMLElement}[1]{\textsf{#1}}

% Abkürzungen
\newcommand{\vgl}{Vgl.\ }
\newcommand{\ua}{\mbox{u.\,a.\ }}
\newcommand{\zB}{\mbox{z.\,B.\ }}
\newcommand{\bs}{$\backslash$}
\newcommand{\quotes}[1]{\glqq #1\grqq} % Anführungszeichen

% Tabelle aus .csv datei erstellen
% aufruf: \perstable{anzahlSpalten}{csvDateipfad}
%die tateien müssen mit komma getrennt sein nicht mit ;

% https://texblog.org/2012/05/30/generate-latex-tables-from-csv-files-excel/
\newcommand{\perstable}[2]{%
	\begin{tabularx}{15cm}{|*{#1}{X|}}
		\hline
		\csvreader[late after line=\\\hline,late after last line=\\\hline,
		no head,column count=#1]
		{#2}
		{}
		{\csvlinetotablerow}
	\end{tabularx}
}

\usepackage{amsmath,amssymb,units}
\usepackage{multicol,wrapfig,caption}
\captionsetup{format=plain,justification=centering,labelsep=newline,singlelinecheck=false,labelfont=bf,font=small}
\usepackage{mathptmx,charter,helvet,courier}
