% ===========================================================================
% Masterlayout
% ===========================================================================
% 	https://www.teuderun.de/latex/layout/satzspiegel/
\documentclass[
	12pt,						% Schriftgröße
	DIV=calc,					% DIV gibt die Satzspiegelgröße  an DIV=10 bei 11pt DIV=calc ist automatisch
	german,						% für Umlaute, Silbentrennung etc.
	paper=a4,					% Papierformat
	oneside,					% oneside= einseitiges Dokument, twoside = zweiseitiges Dokument
	titlepage,					% es wird eine Titelseite verwendet
	parskip=half,				% Abstand zwischen Absätzen (halbe Zeile)
	headings=normal,			% Größe der Überschriften verkleinern
	%lists=totoc,				% Verzeichnisse im Inhaltsverzeichnis aufführen
	listof=totoc,				% Verzeichnisse im Inhaltsverzeichnis aufführen
	bibliography=totoc,			% Literaturverzeichnis im Inhaltsverzeichnis aufführen
	%idx=totoc,					% Index im Inhaltsverzeichnis aufführen
	%captions=tableabove,		% Beschriftung von Tabellen oberhalb ausgeben
	headheight=30pt,
	final,						% Status des Dokuments (final/draft)
	numbers=noenddot			% Punkte am Ende von automatischen Nummerierungen entfernen
]{scrreprt}

\pdfminorversion=7 	% create pdf version 1.7
% ===========================================================================
% Masterlayout ende
% ===========================================================================


% ===========================================================================
% Seitenlayout
% ===========================================================================
% Anpassung des Seitenlayouts ----------------------------------------------
% 	siehe Seitenstil.tex
% --------------------------------------------------------------------------
\usepackage[
automark,			% Kapitelangaben in Kopfzeile automatisch erstellen
headsepline,		% Trennlinie unter Kopfzeile
ilines				% Trennlinie linksbündig ausrichten
]{scrlayer-scrpage}
% ===========================================================================
% Seitenlayout ende
% ===========================================================================


% ===========================================================================
% Einstellungen Zeilenumbrüche
% ===========================================================================
% 	Passe schriftbreite 2% an, um zeilenumbrüche besser zu trennen
\usepackage{microtype}
% fügt so viel Leerraum zwischen den Wörtern ein wie notwendig ist, dass die Zeilenumbrücke sauber sind.
% das kann zu underfull-box arnungen führen
\emergencystretch=1em
% ===========================================================================
% Einstellungen Zeilenumbrüche ende
% ===========================================================================


% Anpassung an Landessprache -----------------------------------------------
% 	Verwendet globale Option german siehe \documentclass
% --------------------------------------------------------------------------
\usepackage[ngerman]{babel}


% Umlaute ------------------------------------------------------------------
% 		Umlaute/Sonderzeichen wie äöüß direkt im Quelltext verwenden (CodePage).
%		Erlaubt automatische Trennung von Worten mit Umlauten.
% --------------------------------------------------------------------------
\usepackage[utf8]{inputenc}
\usepackage[T1]{fontenc}
\usepackage{ae} % "schöneres" ä
\usepackage{textcomp} % Euro-Zeichen etc.


% ===========================================================================
% Grafiken
% ===========================================================================
% Grafiken -----------------------------------------------------------------
% 		Einbinden von Grafiken [draft oder final]
% 		Option [draft] bindet Bilder nicht ein - auch globale Option
% --------------------------------------------------------------------------
\usepackage[dvips,final]{graphicx}
\graphicspath{{Bilder-Screenshots/}} % Dort liegen die Bilder des Dokuments

% binde vectorgrafiken als pdf ein
\usepackage{pdfpages}

% Zum Umfließen von Bildern -------------------------------------------------
\usepackage{floatflt}

% Bilder werden nun wir im Quelltext unter der passenden section angezeigt --
\usepackage[section]{placeins}
% ===========================================================================
% Grafiken ende
% ===========================================================================


% Befehle aus AMSTeX für mathematische Symbole z.B. \boldsymbol \mathbb ----
\usepackage{amsmath,amsfonts,units}

% Einfache Definition der Zeilenabstände und Seitenränder etc. -------------
\usepackage{setspace}
\usepackage{geometry}


% Symbolverzeichnis --------------------------------------------------------
% 	Symbolverzeichnisse bequem erstellen, beruht auf MakeIndex.
% 		makeindex.exe %Name%.nlo -s nomencl.ist -o %Name%.nls
% 		erzeugt dann das Verzeichnis. Dieser Befehl kann z.B. im TeXnicCenter
%		als Postprozessor eingetragen werden, damit er nicht ständig manuell
%		ausgeführt werden muss.
%		Die Definitionen sind ausgegliedert in die Datei Abkuerzungen.tex.
% --------------------------------------------------------------------------
\usepackage[intoc]{nomencl}
  \let\abbrev\nomenclature
  \renewcommand{\nomname}{Abkürzungsverzeichnis}
  \setlength{\nomlabelwidth}{.25\hsize}
  \renewcommand{\nomlabel}[1]{#1 \dotfill}
  \setlength{\nomitemsep}{-\parsep}


% ===========================================================================
% Programmcode Listings
% ===========================================================================
\usepackage{listings}

% see https://www.overleaf.com/learn/latex/Code_Highlighting_with_minted
% to use:
%
% ==BEISPIEL ANFANG==
%\begin{listing}[h]
%	\caption{Python Test Code} 
%	\label{lst:the-code2}
%	\begin{minted}[gobble=2] % entferne die Einrückung von x
%		{python}
%		print("hello, world")
%	\end{minted}
%\end{listing}
% --------------------------------------------------------------------------
% um Dateien zu importieren:
%\begin{listing}
%	\caption{Aufbau Initialisierungsdatei}
%	\label{lst:aufbauInitialisierungsdatei}
%	\inputminted[
%		firstline=191,
%		lastline=239,
%	]{python}
%		{Listings/mail.py}
%\end{listing}
% um lange Dateien zu importieren (mit Seitenumbruch)
%\begingroup
%\captionsetup{type=listing}
%\inputminted[
%	firstline=191,
%	lastline=239
%]{python}
%{Listings/mail.py}
%\captionof{listing}{Facade\label{lst:mailFacade}}
%\endgroup
% ==BEISPIEL ENDE==
\usepackage[
	cache=false,	% cache kommt oft zu fehlern
	chapter			% Listing Nummern Beginnen mit Kapitelnummern
	]{minted}
	
% mögliche styles mit "$ pygmentize -L styles" in Kommandozeile
\usemintedstyle{default}

\definecolor{mintedbackground}{rgb}{0.95,0.95,0.95}

\setminted{
	breaklines, 			% Breche zu lamge Zeilen um
	breakafter=d,			% Breche zu lange zeilen um
	frame=single,			% Kasten um den Code optionen: (none | leftline | topline | bottomline | lines | single)
	framesep=2mm,			% Abstand zwischen oberer und unterer Trennlinie
	baselinestretch=1.2,	% Zeilenabstand
	tabsize=4, 				% leerzeichen für Tab.
	bgcolor=mintedbackground,		% Hintergrundfarbe
	fontsize=\footnotesize,	% Schriftgröße
	linenos=true,			% Aktiviere Zeilennummern
}

% mache autoref auch für mind verfügbar
\providecommand*{\listingautorefname}{Listing}

% https://github.com/gpoore/minted/issues/22
\usepackage{scrhack}

% requirements for minted
\usepackage{keyval}
\usepackage{kvoptions}
\usepackage{fancyvrb}
\usepackage{fvextra}
\usepackage{upquote}
\usepackage{float}
\usepackage{ifthen}
\usepackage{calc}
\usepackage{ifplatform}
\usepackage{pdftexcmds}
\usepackage{etoolbox}
\usepackage{xstring}
\usepackage{xcolor}
\usepackage{lineno}
\usepackage{framed}
\usepackage{shellesc}

% ===========================================================================
% Programmcode Listings
% ===========================================================================

% Lange URLs umbrechen etc. -------------------------------------------------
\usepackage{url}

% Wichtig für korrekte Zitierweise ------------------------------------------
%\usepackage[square]{natbib}
% Quellenangaben in eckige Klammern fassen ----------------------------------
%\bibpunct{[}{]}{;}{a}{}{,~}


% ===========================================================================
% BiBteX
% ===========================================================================
% 	um ein automatisch generiertes Literaturverzeichnis zu erstelen
% 	https://golatex.de/wichtige-hinweise-erstellung-von-literaturverzeichnissen-t11964.html
\usepackage[
	style=alphabetic,
	backend=biber
	]{biblatex}
\usepackage{csquotes}

% Wertebereich von 0 - 10000. Entscheidungskriterium wann links umgebrochen werden.
% value = 0 (Default) -> es werden alle umbrüche gesperrt
% value > 0 -> es wird nach ales Zahlen/Nummern umgebrochen
% value = 9000 -> es wird nach silbentrennung umgebrochen
% https://texwelt.de/fragen/7008/zeilenumbruche-in-bibliografielinks
\setcounter{biburllcpenalty}{9000}% Kleinbuchstaben
\setcounter{biburlucpenalty}{9000}% Großbuchstaben

% ===========================================================================
% BiBteX ende
% ===========================================================================

% Abbildungsverzeichnis
\usepackage{acronym}

% Für Index-Ausgabe; \printindex -------------------------------------------
\usepackage{makeidx}

% ===========================================================================
% PDF-Optionen
% ===========================================================================
% inhaltsverzeichnis konfigurieren
\usepackage[
	bookmarks,
	bookmarksopen=true,
	pdftitle={\titel},
	pdfauthor={\autor},
	pdfcreator={\autor},
	pdfsubject={\titel},
	pdfkeywords={\titel},
	colorlinks=true,
	linkcolor=black, % einfache interne Verknüpfungen
	anchorcolor=black,% Ankertext
	citecolor=blue, % Verweise auf Literaturverzeichniseinträge im Text
	filecolor=magenta, % Verknüpfungen, die lokale Dateien öffnen
	menucolor=red, % Acrobat-Menüpunkte
	urlcolor=cyan, 
	% für die Druckversion können die Farben ausgeschaltet werden:
	%linkcolor=black, % einfache interne Verknüpfungen
	%anchorcolor=black,% Ankertext
	%citecolor=black, % Verweise auf Literaturverzeichniseinträge im Text
	%filecolor=black, % Verknüpfungen, die lokale Dateien öffnen
	%menucolor=black, % Acrobat-Menüpunkte
	%urlcolor=black, 
	%backref,
	%pagebackref,
	plainpages=false,% zur korrekten Erstellung der Bookmarks
	pdfpagelabels,% zur korrekten Erstellung der Bookmarks
	hypertexnames=false,% zur korrekten Erstellung der Bookmarks
	%linktocpage % Seitenzahlen anstatt Text im Inhaltsverzeichnis verlinken
]{hyperref}
% ===========================================================================
% PDF-Optionen ende
% ===========================================================================

% Zum besseren referenzieren auf----------------- ---------------------------
\usepackage[german]{varioref}
\def\myref#1{\autoref{#1}\vpageref[]{#1}}


% Zum fortlaufenden Durchnummerieren der Fußnoten ---------------------------
\usepackage{chngcntr}


% für lange Tabellen
\usepackage{longtable}
\usepackage{array}
\usepackage{ragged2e}
\usepackage{lscape}

% Spaltendefinition rechtsbündig mit definierter Breite ---------------------
\newcolumntype{d}[1]{>{\raggedleft\hspace{0pt}}p{#1}}

% Formatierung von Listen ändern
\usepackage{paralist}
% Standardeinstellungen:
%\setdefaultleftmargin{2.5em}{2.2em}{1.87em}{1.7em}{1em}{1em}


% ===========================================================================
% Diagrammee
% ===========================================================================
% Um diagramme zu drucken ---------------------------------------------------
% https://github.com/MartinThoma/LaTeX-examples/blob/master/tikz/csv-line-plot-two-axes/csv-line-plot-two-axes.tex
% http://pgfplots.sourceforge.net/gallery.html
% http://pgfplots.sourceforge.net/
\usepackage{pgfplots}
\pgfplotsset{
	width=1.0\textwidth,
	height=1.0\textwidth,
	axis lines=center,
	}
\pgfplotsset{compat=1.11}


\usepackage{tikz}

% Package fuer CSV-Dateien --------------------------------------------------
% https://texblog.org/2012/05/30/generate-latex-tables-from-csv-files-excel/
\usepackage{csvsimple} 
\usepackage{tabularx} %für X-Spalten und tabularx-Umgebung
% ===========================================================================
% Diagramme ende
% ===========================================================================

